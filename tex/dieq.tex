\documentclass[10pt, a4paper]{article}
\usepackage[a4paper, total={6in, 8in}]{geometry}
\usepackage[utf8]{inputenc}
\usepackage[T1, T2A]{fontenc}
\usepackage[english,russian]{babel}
\usepackage{amsmath,amsfonts,amssymb}
\usepackage{hyperref}
\setlength\parindent{24pt}

\title{Дифференциальные уравнения}
\author{}
\date{}


\begin{document}

\maketitle

\tableofcontents

\newpage

\section{Уравнения с разделяющимися переменными}
Общий вид:
\par $f(x)dx = g(y)dy$.

\subsection{Метод разделения переменных}
\textbf{Пример:}
\par $xy' - y = 0$
\begin{enumerate}
    \item \textbf{Записать $y'$ как $\frac{dy}{dx}$}.
        \par $x\frac{dy}{dx} - y = 0$
    \item \textbf{В одной стороне собрать все $x$, а в другой - все} $y$.
        \par $\frac{1}{x}dx = \frac{1}{y}dy$
    \item \textbf{Проинтегрировать}.
        \par $\int\frac{1}{x}dx = \int\frac{1}{y}dy$
        \par $\ln|x| + C_{1} = \ln|y| + C_{2}$
        \par $\ln|y| = \ln|x| + C_{3}$
    \item \textbf{Выразить $y$ и записать ответ}.
        \par $e^{\ln|y|} = e^{\ln|x| + C_{3}}$
        \par $|y| = |x| \cdot e^{C_{3}}$
        \par $|y| = |x| \cdot C_{4}$
        \par $y = x \cdot C_{5}$
        \par \textit{Ответ: $y = Cx,\;C \in \mathbb{R}$.}
\end{enumerate}


\section{Неоднородные линейные дифф. ур. 1-го порядка}
Так называемый "НЛДУ1П". \\
Общий вид:
\par $y' + p(x) \cdot y = q(x)$, где
\par $q(x)$ обеспечивает неоднородность,
\par $p(x), q(x)$ -- заданные функции,
\par $y = y(x)$ -- искомая функция.

\subsection{Метод вариации произвольных констант}
\textbf{Пример:}
\par $y' - y \cdot \ctg{x} - \sin{x} = 0$
\begin{enumerate}
    \item Убедиться, что это действительно НЛДУ1П.
        \par $y' + (-\ctg{x}) \cdot y = \sin{x}$
    \item Записать соответствующее однородное дифф. ур. (вместо $q(x)$ записать $0$).
        \par $y' - y \cdot \ctg{x} = 0$
    \item Решить его (методом разделения переменных).
        \par $[...]$
        \par $y = C \cdot \sin{x}$
    \item Записать фразу "Будем искать решение исходного уравнения в виде $y = ...C(x)...$".
        \par\textit{Будем искать решение исходного уравнения в виде $y = ...C(x)...$}
    \item Эту штуку (выражение справа) подставить в исходное уравнение.
        \par $(C(x)\sin{x})' - C(x)\cos{x} - \sin{x} = 0$
        \par $(C(x)\sin{x})' - C(x)\sin{x} \cdot \ctg{x} - \sin{x} = 0$
        \par $C'(x)\sin{x} + C(x)\cos{x} - C(x)\cos{x} - \sin{x} = 0$
        \par $C'(x)\sin{x} = \sin{x}$ (на этом шаге C(x) обязательно должен пропасть)
        \par $C'(x) = 1$
        \par $C(x) = x + C_{1}$
    \item Заглянуть в пункт 4 и записать ответ.
        \par\textit{Ответ: $y = (x + C)\sin{x},\;C \in \mathbb{R}$.}
\end{enumerate}

\subsection{Метод Бернулли}
\textbf{Пример:}
\par $y' - y \cdot \ctg{x} - \sin{x} = 0$
\begin{enumerate}
    \item Убедиться, что это действительно НЛДУ1П.
        \par $y' + (-\ctg{x}) \cdot y = \sin{x}$
    \item Заменить $y = uv$, где $u = u(x)$, $v = v(x)$ -- пока неизвестные функции.
        \par $(uv)' + (\tg{x})(uv) = \frac{1}{\cos{x}}$
        \par $u'v + uv' + uv \cdot tg{x} = \frac{1}{\cos{x}}$
    \item Вынести $u$ за скобки.
        \par $u'v + u(v' + v\tg{x}) = \frac{1}{\cos{x}}$
    \item Найти какую-то конкретную функцию $v \neq 0$, которая обнуляет выражение в скобках.
        \par \textit{Решим $v' + v\tg{x} = 0$}
        \par $[...]$
        \par $v = C_{1}\cos{x}$
        \par \textit{Берём конкретное решение при $C_{1} = 0\;v = \cos{x}$}
    \item Подставить подобранную функцию $v$ в уравнение из пункта 3.
        \par $u'\cos{x} = \frac{1}{\cos{x}}$
        \par $u = \tg{x} + C$
    \item Заглянув в пункт 2, записать ответ.
        \par\textit{Ответ: $y = C\cos{x} + \sin{x}$.}
\end{enumerate}


\section{Уравнение Бернулли}
Общий вид:
\par $y' + p(x) \cdot y = q(x) \cdot y^b$, где
\par $b$ -- число, $b \neq 0$

\subsection{Метод замены $y = z^a$}
\textbf{Пример:}
\par $xy' = \frac{x}{y} + y$
\begin{enumerate}
    \item Убедиться, что это уравнение Бернулли.
        \par $y' - \frac{1}{x}y = y^{-1}$
    \item Заменить $y = z^a$, где $a$ -- пока неизвестное число, $z = z(x)$ -- пока неизвестная функция.
        \par $(z^a)' - \frac{1}{x}z^a = (z^a)^{-1}$
        \par\fbox{$(z(x)^a)'_{x} \Rightarrow (z^a)' = az^{a - 1} \cdot z'$}
        \par $az^{a - 1} \cdot z' - \frac{1}{x}z^a = z^{-a}$
    \item Поделить обе части уравнения на выражение, стоящее перед $z'$.
        \par $z' - \frac{1}{ax}z = \frac{1}{a}z^{-2a + 1}$
    \item Подобрать число $a$ так, чтобы в правой части было $z^0$.
        \par $-2a + 1 = 0 \Rightarrow a = 1/2$
    \item Подобранное $a$ подставить в уравнение из пункта 3.
        \par $z' - \frac{2}{x}z = 2$
    \item Решить это уравнение (НЛДУ1П).
        \par $[...]$
        \par $z = C_{7}x^2-2x$
    \item Заглянув в пункт 2, написать ответ.
        \par\textit{Ответ: $y = \sqrt{Cx^2 - 2x}$.}
\end{enumerate}


\section*{Уравнения, допускающие понижение порядка}
Порядок -- это наибольший встречающийся порядок производной в уравнении. \\
Например: \\
$y' - \frac{1}{x}y = \frac{1}{x^2}$ -- дифф. ур. 1-го порядка \\
$y'' - \sin{(e^{y'''} + \frac{4}{y''})} = \frac{1}{\sqrt{y}}$ -- дифф. ур. 3-го порядка

\section{Дифф. уравнения вида $y^{n} = f(x)$}
\subsection{Метод n-кратного интегрирования}
\textbf{Пример:}
    \par $y''' = 3x^2 + 6$
    \par $y''  = x^3 + 6x + C_{1}$
    \par $y'   = \frac{x^4}{4} + 3x^2 + C_{1}x + C_{2}$
    \par $y    = \frac{x^5}{20} + x^3 + C_{4}x^2 + C_{2}x + C_{3}$
    \par\textit{Ответ: $y = \frac{x^5}{20} + x^3 + C_{4}x^2 + C_{2}x + C_{3}$}

\section{Дифф. уравнения, в которых нет $y$ без $'$ и которые содержат как минимум две производные}
\subsection{Метод замены $z = z(x) = y^{\min}$}
\textbf{Пример:}
\par $y'''' - y''' = e^x$
\begin{enumerate}
    \item Заменить $z = z(x) = y^{(...)}$, где $(...)$ — наименьший встречающийся порядок производной.
        \par\textit{Замена $z = z(x) = y'''$}
    \item Переписать уравнение через $z$.
        \par $z' - z = e^x$
        \par $[...]$
        \par $z = (x + C)e^x$
    \item Заменить обратно.
        \par $y''' = e^x(x + C)$
        \par\textit{Решим методом n-кратного интегрирования}
        \par $[...]$
        \par $y = e^x(x + C - 3) + C_{7} + C_{2}x + C_{8}x^2$
        \par\textit{Ответ: $y = e^x(x + C - 3) + C_{7} + C_{2}x + C_{8}x^2$.}
\end{enumerate}

\section{Дифф. уравнения, в которых нет $x$}
\textbf{Пример:}
\par $y'' \cdot y = y'^{2}$
\subsection{Метод замены $z = z(y) = y'$}
\begin{enumerate}
    \item Заменить $z = z(y) = y'$.
        \par\textit{Введём новую функцию $z = z(y) = y'$}
    \item Выразить через $z$ все производные, встречающиеся в уравнении.
        \par $y' = z$
        \par $y'' = y''_{xx} = (y'_{x})'_{x} = \frac{d(y'_{x})}{dx}  \cdot  \frac{dy}{dx} = \frac{dz}{dy}  \cdot  \frac{dy}{dx} = z' \cdot y' = z' \cdot z$
        \par\textit{Имеем: $z' \cdot z \cdot y = z^2$ (искомая функция $z = z(y)$)}
        \par $z' \cdot y = z$
        \par $\frac{dz}{dy} \cdot y = z$
        \par $\frac{dz}{z} = \frac{dy}{y}$
        \par $\ln|z| = \ln|y| + C_{1}$
        \par $z = C_{2}y$
    \item Заменить обратно.
        \par $y' = C_{2}y$
        \par $\frac{dy}{dx} = C_{2}y$
        \par $\ln|y| = C_{2}x + C_{3}$
        \par $y = e^{C_2{x}}  \cdot  C_4$
        \par\textit{Ответ: $y = C_1e^{C_2x},\;C_1,\;C_2 \in \mathbb{R}$.}
\end{enumerate}


\section{Уравнение Лагранжа}
Общий вид уравнения Лагранжа:
\par $y = x \cdot F(y') + G(y')$, где
\par $F(y'), G(y')$ -- выражения, в которых нет $y,\;y'',\;y'''$ и т. д.

\subsection{Метод замены $p = p(x) = y'$}
\textbf{Пример уравнение Лагранжа, не являющегося уравнением Клеро:}
\par $y + 3y'^2 = 2xy'$ $(*)$
\begin{enumerate}
    \item Убедиться, что это уравнение Лагранжа, не являющееся уравнением Клеро.
        \par $t = x \cdot 2y' + (-3y'^2)$
    \item Заменить $p = p(x) = y'$.
        \par\textit{Замена $p = p(x) = y' \Rightarrow y = 2px - 3p^2$ $(**)$}
    \item Продифференцировать обе части.
        \par $y' = 2p'x + 2p - 6pp'$
    \item Заменить $y'$ на $p$.
        \par $p = 2p'x + 2p - 6pp'$
    \item Все слагаемые с $p'$ перенести влево, вынести за скобку $p'$, а всё остальное — вправо.
        \par $(6p - 2x)p' = p$
    \item Переписать $p'$ как $\frac{dp}{dx}$ и умножить обе части на $\frac{dx}{dp}$.
        \par $(6p - 2x)\frac{dp}{dx} = p$
        \par $6p - 2x = p\frac{dx}{dp}$
        \par\textit{$6p - 2x = px'$ (это НЛДУ1П, искомая функция $x = x(p)$)}
    \item Рассмотреть случай (множитель при $x'$) $= 0$.
        \begin{enumerate}
            \item[(7а)] $p = 0$ \\
                  $y' = 0$ \\
                  $y = A$
            \item[(7б)] Полученное выражение подставить в $(*)$. \\
                  $A + 3 \cdot 0^2 = 2x \cdot 0$ \\
                  $A = 0$
            \item[(7в)] Первое решение дифф. уравнения. \\
                  $y = 0$
        \end{enumerate}
    \item Рассмотреть случай (множитель при $x'$) $\neq 0$.
        \begin{enumerate}
            \item[(8а)] Поделить на $p'$. \\
                  $6 - \frac{2x}{p} = x'$ \\
                  $x' + \frac{2}{p}x = 6$ \\
                  $[...]$ \\
                  $x = 2p + \frac{B}{p^2}$
            \item[(8б)] Полученное выражение для $x$ подставим в $(**)$. \\
                  $y = 2p  \cdot  (2p + \frac{B}{p^2}) - 3p^2 = p^2 + \frac{2B}{p}$
        \end{enumerate}
    \item Записать ответ,
        \begin{itemize}
            \item учитывая результат 7в,
            \item учитывая результат 8а и 8б,
            \item заменяя константы на $C$,
            \item заменяя $p$ на $x$.
        \end{itemize}
        \par\textit{Ответ:}
              $\left[
                  \begin{gathered}
                      y = 0, \\
                      \begin{cases}
                          x = 2t + \frac{C}{t^2}, \\
                          y = t^2 + \frac{2C}{t},\;C \in \mathbb{R}
                      \end{cases}
                  \end{gathered}
              \right.$.
\end{enumerate}

\section{Уравнение Клеро}
Уравнение Клеро -- частный случай уравнения Лагранжа. \\
Общий вид уравнения Клеро:
\par $y = xy' + G(y')$

\subsection{Метод замены $p = p(x) = y'$}
\textbf{Пример:}
\par $y - y'^2 = xy'$ $(***)$
\begin{enumerate}
    \item[1--5.] Шаги идентичны предыдущему методу.
    \item[6.] Рассмотреть случай (множитель при $x'$) $= 0$.
        \begin{itemize}
            \item[(6а)] $p' = 0$ \\
                        $y'' = 0$ \\
                        $y = Ax + B$
            \item[(6б)] Подставить выражения в $(***)$ \\
                        $Ax + B - A^2 = Ax$
            \item[(6в)] Выразить обе константы через какую-то одну \\
                        $\begin{cases}
                            A = A \\
                            B = A^2
                        \end{cases}$
            \item[(6г)] Записать первую серию решений уравнения \\
                        $y = Ax + A^2$
        \end{itemize}
    \item[7.] Рассмотреть случай (множитель при $x'$) $\neq 0$.
        \begin{itemize}
            \item[(7а)] $x + 2p = 0$
                        $x = -2p$
            \item[(7б)] Подставим выражение в пункт 2
                        $y = xp + p^2$
                        $y = (-2)p + p^2$
                        $y = -p^2$
        \end{itemize}
    \item[8.]   Записать ответ
        \begin{itemize}
            \item учитывая результат 6г,
            \item учитывая результат 7а и 7б,
            \item заменяя константы на $C$,
            \item заменяя $p$ на $t$.
        \end{itemize}
        \par\textit{Ответ:}
            $\left[
              \begin{gathered}
                  y = Cx + C^2,\;C \in \mathbb{R} \\
                  \begin{cases}
                      x = -2t, \\
                      y = -t^2
                  \end{cases}
              \end{gathered}
          \right.
          \Leftrightarrow
          \left[
              \begin{gathered}
                  y = Cx + C^2,\;C \in \mathbb{R} \\
                  y = -\frac{x^2}{4}
              \end{gathered}
          \right.$.
\end{enumerate}


\section*{Уравнения, не разрешённые относительно $y'$}
Уравнение решено относительно $y'$ $\Leftrightarrow$ $y'$ выражено через всё остальное \\
Например: \\
$\sin{x + y'} = e^{xy'}$ -- не решено относительно $y'$.

\section{Уравнение вида $x = F(y')$}
\subsection{Метод замены $p = y'$}
\textbf{Пример:}
\par $\ln{y'} + \sin{y'} - x = 0$
\begin{enumerate}
    \item Убедиться, что это уравнение вида $x = F(y')$.
        \par $x = \ln{y'} + \sin{y'}$
    \item Заменить $p = y'$.
        \par $x = \ln{p} + \sin{p}$
    \item Взять дифференциал от обеих частей.
        \par $dx = d(\ln{p} + sin{p})$
        \par $dx = (\frac{1}{p} + \cos{p})dp$ $(*)$
    \item Из замены выразить $dx$ через $dy$.
        \par $p = y'$
        \par $p = \frac{dy}{dx}$
        \par $dx = \frac{dy}{p}$ $(**)$
    \item Подставить $(**)$ в $(*)$.
        \par $\frac{dy}{p} = (\frac{1}{p} + \cos{p})dp$
    \item Выразить $dy$ и проинтегрировать.
        \par $dy = p(1 + \cos{p})dp$
        \par $y = \int{(1 + p\cos{p}dp} = p + C_1 + \int{p\cos{p}dp} = p + p\sin{p} \cos{p} + C$
    \item Записать ответ
        $\begin{cases}
            p \rightarrow t \\
            const \rightarrow C
        \end{cases}$.
        \par\textit{Ответ:}
            $\begin{cases}
                x = \ln{t} + \sin{t} \\
                y = t + t\sin{t} + \cos{t} + C,\;C \in \mathbb{R}
            \end{cases}$.
\end{enumerate}

\section{Уравнение вида $y = F(y')$}
\subsection{Метод замены $p = y'$}
\textbf{Пример:}
\par $e^{\ln{y} - y'} = y'^2$
\begin{enumerate}
    \item Убедиться, что это уравнение вида $y = F(y')$.
        \par $\frac{y}{e^{y'}} = y'^2$
        \par $y = y'^2e^{y'}$
    \item Заменить $p = y'$.
        \par $y = p^2e^p$
    \item Взять дифференциал от обеих частей.
        \par $dy = (2pe^p + p^2e^p)dp$ $(*)$
    \item Из замены выразить $dy$ через $dx$.
        \par $p = y'$
        \par $p = \frac{dy}{dx}$
        \par $dy = pdx$ $(**)$
    \item Подставить $(**)$ в $(*)$.
        \par $pdx = (2pe^p + p^2e^p)dp$
    \item Выразить $dx$ и проинтегрировать.
        \par $dx = \frac{2pe^p + pe^2}{p}dp$
        \par $dx = (2e^p + pe^p)dp$
        \par $x = 2\int{e^pdp} + \int{pe^pdp} = (2e^p + A) + (pe^p - e^p + B) = e^p + pe^p + C$
        \par $\int{pe^pdp} = pe^p - \int{e^pdp} = pe^p - e^p + B$
    \item Записать ответ
        $\begin{cases}
            p \rightarrow t \\
            const \rightarrow C
        \end{cases}$.
        \par\textit{Ответ:}
        $\begin{cases}
            y = t^2e^t, \\
            x = te^t - e^t + C,\;C \in \mathbb{R}
        \end{cases}$.
\end{enumerate}


\section{Однородные дифф. уравнения}
\textbf{Опред.} Однородным называется уравнение вида $y' = F(\frac{y}{x})$. \\
Не путать с НЛДУ1П -- уравнением вида $y' + p(x)y = q(x)$

\subsection{Метод замены $y = x \cdot u$}
\textbf{Пример:}
\par $xy' - y - xe^{\frac{y}{x}} = 0$
\begin{enumerate}
    \item Убедиться, что это уравнение вида $y' = F(\frac{y}{x})$.
        \par $xy' = xe^{\frac{y}{x}} + y$
        \par $y' = e^{\frac{y}{x}} + \frac{y}{x}$
    \item Заменить $y = x \cdot u$, где $u = u(x)$.
        \par\textit{Замена $y = x \cdot u$}
        \par $(xu)' = e^u + u$
        \par $u + xu' = e^u + u$
        \par $xu' = e^u$
    \item Решить полученное уравнение относительно $u = u(x)$ (методом разделения переменных).
        \par $x\frac{du}{dx} = e^u$
        \par $\frac{1}{e^u}du = \frac{1}{x}dx$
        \par $\int{\frac{1}{e^u}du} = \int{\frac{1}{x}dx}$
        \par $-\frac{1}{e^u} = \ln{|x|} + C_1$
        \par $e^{-u} = C_2 - \ln{|x|}$
        \par $\ln{e^{-u}} = \ln{(C_2 - \ln{|x|)}}$
        \par $u = -\ln{C_2 - \ln{|x|}}$
    \item Заменить обратно и записать ответ.
        \par\textit{Ответ: $y = -x\ln{(C - \ln{|x|})},\;C \in \mathbb{R}$.}
\end{enumerate}


\section{Уравнение вида  $y' = F(\frac{a_1x + b_1y + c_1}{a_2x + b_2y + c_2})$}
Данный тип уравнения делится на 2 подтипа:
\begin{enumerate}
    \item[(12а)] Если $a_1x + b_1y$ и $a_2x + b_2y$ пропорциональны.
    \item[(12б)] Если $a_1x + b_1y$ и $a_2x + b_2y$ непропорциональны.
\end{enumerate}

\subsection{Метод решения подтипа 12а}
\textbf{Пример:}
\par $(x - 2y + 3)dx = (-2x  + 4y + 8)dy$
\begin{enumerate}
    \item Убедиться, что это уравнение подтипа 12а.
        \par $y' = \frac{x - 2y + 3}{-2x + 4y + 8}$ $(*)$
        \par $F(t) = t$
        \par \begin{tabular}{c c c}
             $a_1 =  1$  & $b_1 = 2$ & $c_1 = 3$  \\
             $a_2 = -2$ & $b_2 = 4$ & $c_2 = 8$
        \end{tabular}
    \item Заменить $z = z(x) =$ \textit{[одному из этих членов $a_1x + b_1y$ или $a_2x + b_2y$]}.
        \par $z = z(x) = x - 2y$
    \item Выразить отсюда $y$ и продифференцировать по $x$.
        \par $y = \frac{x}{2} - \frac{z}{2}$
        \par $y' = \frac{1}{2} - \frac{z'}{2}$
    \item В ур-и $(*)$ из пункта 1 заменить левую и правую часть.
        \par\textit{$\frac{1}{2} - \frac{z'}{2} = \frac{z + 3}{-2z + 8}$ -- искомая функция $z = z(x)$}
    \item Решить полученное ДУ (заведомо пройдёт метод разделения переменных).
        \par $[...]$
        \par\textit{$\frac{1}{4}(2z - 1 - 7\ln{|2z - 1|}) = x + C_1$ -- решение в неявном виде}
    \item Заменить обратно и записать ответ.
        \par $\frac{1}{4}(2x - 4y - 1 - 7\ln{|2x - 4y - 1|}) = x + C_1$
        \par\textit{Ответ: $\frac{x}{2} + y + \frac{7}{4}\ln{|2x - 4y - 1|} = C,\;C \in \mathbb{R}$.}
\end{enumerate}

\subsection{Метод решения подтипа 12б}
\textbf{Пример:}
\par $(x - y - 1)dx + (x + 2y - 4)dy = 0$
\begin{enumerate}
    \item Убедиться, что это уравнение подтипа 12б.
        \par $y' = \frac{x - y - 1}{-x - 2y + 4}$ $(*)$
        \par $F(t) = t$
        \par \begin{tabular}{c c c}
             $a_1 =  1$ & $b_1 = -1$ & $c_1 = 1$  \\
             $a_2 = -1$ & $b_2 = -2$ & $c_2 = 4$
        \end{tabular}
    \item Записать и решить систему $\begin{cases}
                                        \textit{Числитель}$ = 0$ \\
                                        \textit{Знаменатель}$ = 0$ \\
                                     \end{cases}$ и обозначить решение через $x_0$ и $y_0$.
        \par$\begin{cases}
                x - y - 1 = 0 \\
                -x - 2y + 4 = 0
            \end{cases}$
            $\begin{cases}
                x = 2 \\
                y = 1
            \end{cases}$
            $\Rightarrow\;x_0 = 2,\;y_0 = 1$
    \item Заменить $\begin{cases}
                        x = u + x_0 \\
                        y = v + y_0
                    \end{cases}$, где $u, v$ -- новые переменные и $v = v(u)$.
        \par$\begin{cases}
                x = u + 2 \\
                y = v + 1
            \end{cases}$
    \item Подставить это в уравнение $(*)$ из пункта 1.
        \par\textit{Левая часть $ = y' =  \frac{dy}{dx} = \frac{d(v + 1)}{d(u + 2)} = \frac{dv}{du} = v'_u$}
        \par\textit{Правая часть $ = \frac{u + 2 - v - 1 - 1}{-u - 2 - 2v - 2 + 4} = \frac{u - v}{-u - 2v}$}
        \par\textbf{Если в числителе и знаменателе не исчезли свободные члены, ищи ошибку!}
        \par\textit{Приходим к ур-ю}
        \par $v' = \frac{u - v}{-u - 2v}$
    \item Решить это ДУ, разделив числитель и знаменатель на $u$ и рассмотрев как уравнение вида $v' = F(\frac{v}{u})$).
        \par $v' = \frac{1 - \frac{v}{u}}{-1 - 2\frac{u}{v}}$
        \par $[...]$
        \par\textit{$...u...v... = ...u...v...$ -- решение в неявном виде}
    \item Заменить обратно и записать ответ.
        \par$\begin{cases}
            u = x - 2 \\
            v = y - 1
            \end{cases}$
        \par\textit{Ответ: $...(x - 2)...(y - 1)... = ...(x - 2)...(y - 1)...$.}
\end{enumerate}


\section{Уравнение в полных дифференциалах}
Дифференциал:
\par $d(u(x, u)) = u'_xdx + u'_ydy$
\par $d(x^2\ln{y}) = 2x\ln{y}dx + \frac{x}{y}dy$ \\
\textbf{Опред.} Уравнением в полных дифференциалах называется уравнение $P(x, y)dx + Q(x, y)dy = 0$,
у которого в левой части стоит дифференциал некоторой функции двух переменных, т. е. $\exists u(x, y):
\begin{cases}
    P(x, y) = u'_x \\
    Q(x, y) = u'_y
\end{cases}$.

\subsection{Метод поиска полного дифференциала}
\textbf{Пример:}
\par $(-x^2 + y\cos{x})dx + (y + \sin{x})dy = 0$
\begin{enumerate}
    \item Убедиться, что это уравнение в полных дифференциалах.
        \par
            $\begin{cases}
                (-x^2 + y\cos{x})'_y = \cos{x} \\
                (y + \sin{x})'_x = \cos{x}
            \end{cases}
            \Rightarrow
            \cos{x} = \cos{x}
            \Rightarrow$
            \textit{это уравнение в полных дифференциалах}
    \item Записать систему $\begin{cases}
                               u'_x = P \\
                               u'_y = Q
                            \end{cases}$.
        \par
            $\begin{cases}
                u'_x = -x^2 + y\cos{x}\;(1) \\
                y'_y = y + \sin{x}\;(2)
            \end{cases}$
    \item Выбрать любое уравнение и проинтегрировать по соответствующей переменной.
        \par $(1)\;\Rightarrow\;u(x, y) = \int{(-x^2 +y\cos{x})}dx = -\frac{x^3}{3} + y\sin{x} + A(y)$
        \par \textit{$A(y)$ -- константа с точки зрения интегрирования по $x$}
    \item Полученное выражение подставить в другое уравнение из системы пункта 2.
        \par $(2)\;\Rightarrow\;(-\frac{x^3}{3} + y\sin{x} + A(y))'_y = y + \sin{x}$
        \par $\sin{x} + A'_y(y) = y + \sin{x}$
        \par $A(y) = \frac{y^2}{2} + C_1$
    \item Записать окончательное выражение для $u(x, y)$.
        \par $u(x, y) = -\frac{x^3}{3} + y\sin{x} + \frac{y^2}{2} + C_1$
    \item Вернуться к истокам.
        \par $u'_xdx + u'_ydy = 0$
        \par $d(u(x, y)) = 0$
        \par $u(x, y) = C_2$
        \par $-\frac{x^3}{3} + y\sin{x} + \frac{y^2}{2} + C_1 = C_2$
        \par\textit{Ответ: $-\frac{x^3}{3} + y\sin{x} + \frac{y^2}{2} = C, C \in \mathbb{R}$}
\end{enumerate}

\subsection{Метод поиска интегрирующего множителя}
В контрольной работе задания на данный метод \underline{не будет}. \\
\textbf{Пример:}
\par $(-\frac{3}{x} - 2y)dx + (\frac{6}{y} - 3x)dy = 0$
\begin{enumerate}
    \item Убедиться, что это уравнение не в полных дифференциалах.
        \par
            $\begin{cases}
                (-\frac{3}{x} - 2y)'_y = -2 \\
                (\frac{6}{y} - 3x)'_x = -3
            \end{cases}
            \Rightarrow
            -2 \neq -3
            \Rightarrow$
            \textit{это не уравнение в полных дифференциалах}
    \item Стоит поискать интегрирующий множитель в виде $x^a \cdot y^b$, где $a$, $b$ -- числа.
        \par $[(-\frac{3}{x} - 2y)dx + (\frac{6}{y} - 3x)dy] \cdot x^ay^b = 0 \cdot x^ay^b$
        \par $(3x^{a - 1}y^{b} - 2x^{a}y^{b + 1})dx + (6x^{a}y^{b - 1} - 3x^{a + 1}y^{b})dy = 0$
    \item Записать условие того, что это уравнение в полных дифференциалах.
        \par
            $\begin{cases}
                (3x^{a - 1}y^{b} - 2x^{a}y^{b + 1})'_y\\
                (6x^{a}y^{b - 1} - 3x^{a + 1}y^{b})'_x
            \end{cases}$
        \par
            $\begin{cases}
                3x^{a - 1} \cdot by^{b - 1} - 2x^{a} \cdot (b + 1)y^{b} \\
                y^{b - 1} \cdot 6ax^{a - 1} - y^{b} \cdot 3(a + 1)x^{a}
            \end{cases}$
        \par
            $\begin{cases}
                3bx^{a - 1}y^{b - 1} - 2(b + 1)x^{a}y^{b} \\
                6ax^{a - 1}y^{b - 1} - 3(a + 1)x^{a}y^{b}
            \end{cases}
            \Rightarrow$
        \par
            $\begin{cases}
                3b = 6a \\
                2(b  + 1) = 3(a + 1)
            \end{cases}
            \Rightarrow
            \begin{cases}
                a = 1 \\
                b = 2
            \end{cases}$
    \item Домножить исходное дифф. ур. на угаданный интегрирующий множитель.
        \par $(3y^2 - 2x^3)dx + (6xy - 3x^2y^2)dy = 0$
        \par \textit{Решаем как уравнение в полных дифференциалах} $[...]$
\end{enumerate}


\section{ОЛДУ с постоянным коэффициентом}

\textbf{Пример:}
    \par $y^{(5)} - 4y^{(4)} + 13y^{'''} = 0$
\subsection{Метод линейных комбинаций}
\begin{enumerate}
    \item Записать характеристическое уравнение:
          $y \rightarrow \lambda$, порядок производной $\rightarrow$ степень.
        \par $\lambda^5 - 4\lambda^4 + 13\lambda^3 = 0$
    \item Разложить левую часть на линейные множители.
        \par $(\lambda - 3)^3(\lambda - 2 - 3i)(\lambda - 2 + 3i) = 0$
    \item Выписать список корней и их кратности.
        \par $\lambda_1 = 0,\;k_1 = 3$
        \par $\lambda_2 = 2 \pm 3i,\;k_2 = 1$
    \item Около каждого корня (или пары комплексных сопряжённых корней) выписать серию функций (или две серии функций):
    \begin{itemize}
        \item $\lambda = \alpha,\;k = m\;\rightarrow\;e^{\alpha x}, \;
              xe^{\alpha x},\;x^2e^{\alpha x}, \dots,\;x^me^{\alpha x}$
        \item $\lambda = \alpha \pm \beta i\;(\alpha \in \mathbb{R}, \beta \in \mathbb{R}, \beta > 0),\;k = m,\;\rightarrow\;$ \\
              $(e^{\alpha x}\cos{\beta x},\;x e^{\alpha x}\cos{\beta x},\;\dots\;, x^m e^{\alpha x}\cos{\beta x}),$ \\
              $(e^{\alpha x}\sin{\beta x},\;x e^{\alpha x}\sin{\beta x},\;\dots\;, x^m e^{\alpha x}\sin{\beta x})$
    \end{itemize}
        \par $\lambda_1 = 0,\;k_1 = 3\;\rightarrow \;
             e^{0x},\;xe^{0x},\;x^2e^{0x}$
        \par $\lambda_2 = 2 \pm 3i,\;k_2 = 1\;\rightarrow \;
             (e^{2x}\cos{3x}),\;(e^{2x}\sin{3x})$
    \item Записать ответ в виде линейное комбинации этих функций.
        \par\textit{Ответ: $y = C_1 + C_2x + C_3x^2 + C_4e^{2x}\cos{3x} + C_4e^{2x}\sin{3x},\;
                   C_1, C_2, C_3, C_4 \in \mathbb{R}$.}
\end{enumerate}


\end{document}
